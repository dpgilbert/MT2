\documentclass{beamer}
\usepackage[absolute,overlay]{textpos}

\geometry{paperwidth=140mm,paperheight=105mm}

\title{Beyond the Chandrasekhar Limit in Carbon-Oxygen White Dwarf Mergers}
\subtitle{Stellar Structure Final Presentation}
\author{Dylan Gilbert}
\institute[University of California, San Diego]

\date{Fall 2016}

%\AtBeginSubsection[]
%{
%\begin{frame}<beamer>{Outline}
%  \tableofcontents[currentsection,currentsubsecton]
%\end{frame}
%}

\begin{document}

\begin{frame}
\titlepage
\end{frame}

\begin{frame}{Kepler supernova remnant (1604), a type Ia (Chandra)}
\begin{center}\includegraphics[scale=0.4]{kepler}\end{center}
\end{frame}

\begin{frame}{Outline}
\tableofcontents
\end{frame}

\section{Introduction}

\subsection{White Dwarf Basics}

\begin{frame}{White Dwarfs (WDs)}{Basics}
  \begin{itemize}
  \item {Compact remnants of most stars ($M< 8M_{\odot}$).}
  \item {Solitary WDs tend to have $M\sim 0.6M_{\odot}$, but binaries can have more variable masses due to mass transfer, e.g. Sirius B, $M=0.978M_{\odot}$.}
  \item {Masses can be read off to high precision from pressure broadening of lines.}
  \item {Supported by electron degeneracy pressure. Unstable above the Chandrasekhar mass, $M_{C} \sim 1.4M_{\odot}$. In a merger of Sirius B-sized WDs, 1+1=2 leads to catastrophic instability.}
  \item {Small radius ($\sim$$R_E$), therefore low luminosity, but high temperature.}     
  \end{itemize}
\end{frame}

\begin{frame}{WDs}{HR Diagram}
\begin{itemize}
\item HR diagram from HW1Q3 of stellar spectral class points with 29 nearest stars plotted.
\includegraphics[scale=0.4]{HRe}
\begin{textblock*}{64mm}(15mm,0.7\textheight)
\begin{exampleblock}{}
  Sirius B
\end{exampleblock}
\end{textblock*}

\begin{textblock*}{64mm}(32mm,0.75\textheight)
\begin{exampleblock}{}
  Procyon B
\end{exampleblock}
\end{textblock*}
\end{itemize}
\end{frame}

\begin{frame}{WDs}{Sirius System}
  \begin{itemize}
  \item {In visible light, Sirius A is much brighter than Sirius B.}

    \includegraphics[scale=0.15]{Siriusvis}

    (Hubble image)
    \end{itemize}
\end{frame}

\begin{frame}{WDs}{Sirius System}
  \begin{itemize}
    \item {In x-rays, Sirius B is much brighter than Sirius A.}
    
      \includegraphics[scale=0.5]{Siriusxray}
     
      (Chandra image)
  \end{itemize}
\end{frame}

\subsection{WD Compositions}

\begin{frame}{WD Compositions}{From Nelemans et al (2001b)}
  \begin{itemize}
    \item {No solitary He WDs yet. Almost all solitary WDs are CO, with 1\% ONe and Si.}
    \item {Mass transfer in binaries, mostly during the red giant phases, allows for He WDs. In fact, most binary WD systems have at least one He WD.}
    \item {He WDs tend to have $M\sim0.4M_{\odot}\rightarrow$ they don't contribute much to super-Chandrasekhar binary pairs.}
    \item {Around 1/4 binary WDs are CO-CO, and around half of these are super-Chandrasekhar.}
  \end{itemize}
\end{frame}

\begin{frame}{WD Binary Evolution}
  \begin{itemize}
    \item {Fig 1 from Nelemans et al. (2001). Clockwise from top left: He-He, CO-CO, He-CO, CO-He.}
    \item {Note that CO-CO pairs require the widest separation to minimize mass transfer, and that $215R_{\odot}=1$ AU.}
  \begin{center} \includegraphics[scale=0.35]{nelemansfig1}\end{center}
  \end{itemize}
\end{frame}

\section{Frequency}

\begin{frame}{How many candidates?}{From Nelemans et al. (2001b)}
  \begin{itemize}
    \item {$10^{10}$ WDs total in Milky Way.}
    \item {Around 2.5\% of WDs are already in a WD binary. About half will be eventually (Raghavan et al, 2010). Many current WDs still have a main sequence companion that will evolve into a WD, and many current main sequence binaries will one day be WD binaries.}
    \item {Not all binaries will merge in a reasonable time frame. Gravitational wave losses in systems with current periods $\sim$12 hours will produce a merger within a Hubble time (about half of {\it current} binaries).}
    \item {Altogether, more than $10^8$ WD binaries will merge within a Hubble time in the Milky Way.}
    \item {$10^8$ in $10^{10}$ years means 1 per 100 years, and about 1 super-Chandrasekhar CO-CO pair per 300 years.}
    \item {Each remains spectacularly visible for around 10,000 years (more on that later), so there should be dozens in both the Milky Way and Andromeda.}
  \end{itemize}
\end{frame}

\section{Observability}

\subsection{Tidal Disruption}

\begin{frame}{Observability}{Tidal Disruption}
  \begin{itemize}
    \item {WDs are small and therefore hard to see, but merger disrupts them and creates a $\sim$0.1pc nebula. Figure 3 from Dan et al (2014) depicts tidal disruption during merger.}
      \begin{center}\includegraphics[scale=0.4]{danfig3}\end{center}
    \item {Especially massive pairs explode in type Ia supernovae.}
  \end{itemize}
\end{frame}

\subsection{HR Diagram}

\begin{frame}{Observability}{Path on HR Diagram for Non-Supernova Mergers (Schwab et al 2016)}
  \begin{itemize}
    \item {Fig 11 from Schwab et al. Note the large effective radii, 10000 $L_{\odot}$.}
    \item {(1) C burning ignites, (2) degeneracy lifts (T skyrockets), (3) T high enough to ionize nebula, (4) neutrino-cooled contraction.}
      \begin{center}\includegraphics[scale=0.55]{schwabfig11}\end{center}
      \item {The emission nebula formed in (3) will look like any other planetary nebula, except for a smoking gun absence of H and He. There should be dozens in the Milky Way and Andromeda, but none are known; perhaps they are obscured by dust produced in the merger.}
  \end{itemize}
\end{frame}

\subsection{Supernovae}

\begin{frame}{Observability}{Supernovae}
  \begin{itemize}
    \item{Merging CO-CO WDs have recently gained favor as a source of type Ia supernovae, but the fractional contributions remain uncertain (Dan et al 2015, Howell et al 2006).}
    \item{Highly super-Chandrasekhar pairs $\sim$2$M_{\odot}$ explode as extra-energetic type Ia supernova. Both WDs tidally disrupt each other and collapse to a single highly super $M_C$ body, instead of steadily ramping one WD's mass to $M_C$ via accretion (Dan et al 2014, Howell et al 2006).}
    \item {Fig 3 from Dan et al. (2014). }
      \includegraphics[scale=0.4]{danfig1}
  \end{itemize}
\begin{textblock*}{64mm}(85mm,0.5\textheight)
\begin{exampleblock}{}
CO-CO mergers are 

on the right
\end{exampleblock}
\end{textblock*}


\end{frame}

\begin{frame}{Observability}{SNLS-03D3bb from Howell et al (2006)}
\begin{itemize}
      \item{SNLS-03D3bb (redshift 0.24) was 2.2x brighter than standard candle type Ia prediction models suggested.}
    \item {Figure 2 from Howell et al (2006) shows type Ia supernovae near enough to have brightness cross checks available. The dotted line is the hard limit for $M_C$ type Ia Ni-56 production.}
\begin{center}\includegraphics[scale=0.35]{howellfig2}\end{center}
\end{itemize}
\end{frame}

\section{Modeling}

\begin{frame}{Modeling}{Novel Contributions from Schwab et al (2016)}
  \begin{itemize}
    \item {Internal changes are not reflected on surface in real time, but impact on final state is.}
    \item {Some super-Chandrasekhar mergers eject enough mass to remain below $M_C$.}
    \item {Can be distinguished from AGB stars via lack of H and He.}
    \item {Include effects of viscosity: C burning starts in hours, not 10,000 years. (Schwab et al, 2012)}
    \item {The evolution of super-Chandrasekhar mergers is characterized by a series of mass cutoffs and, in stable mergers, repeated ignition of off-center burning (next slide).}
\end{itemize}
\end{frame}

\begin{frame}{Modeling}{Schwab et al (2016) continued: Mass Cutoffs}
\begin{itemize}
    \item {Higher mass cores contract more efficiently via neutrino cooling and so ignite more easily.}
    \item {If an ONe core cools to degeneracy, it collapses via e-capture at $1.38M_{\odot}$. But, a post-merger WD has degeneracy lifted by C burning, and hot ONe WDs ignite below $1.38M_{\odot}$ (even 50/50 ONe).}
    \item {If Si burns, WD$\rightarrow$NS.}
    \item {Fig C2 from Schwab et al (2016). }
\end{itemize}
    \begin{center}\includegraphics[scale=1.]{schwabfigc2}\end{center}
\begin{textblock*}{64mm}(44mm,0.55\textheight)
\begin{exampleblock}{}
  1.35$M_{\odot}$ 
\end{exampleblock}
\end{textblock*}

\begin{textblock*}{64mm}(58mm,0.55\textheight)
\begin{exampleblock}{}
  1.375$M_{\odot}$ (\underline{Not} EC)
\end{exampleblock}
\end{textblock*}

\begin{textblock*}{64mm}(84mm,0.48\textheight)
\begin{exampleblock}{}
  1.41$M_{\odot}$ 
\end{exampleblock}
\end{textblock*}
\end{frame}

\begin{frame}{Modeling}{WD Composition Progression from Schwab et al (2016)}
\begin{itemize}
\item {Fig 3 from Schwab et al (2016). Note especially that the ONe stage of the merger is closer to the Ne case than 50-50 Ne/O case from the previous slide. }
\begin{center}\includegraphics[scale=0.5]{schwabfig10}\end{center}
\begin{textblock*}{64mm}(95mm,0.56\textheight)
\begin{exampleblock}{}
  2:1 Ne:O, roughly.
\end{exampleblock}
\end{textblock*}

\end{itemize}
\end{frame}

\section{Summary}

\begin{frame}{Summary}
\begin{itemize}
\item {Dozens of super-Chandrasekhar WD merger remnants should be visible in nearby galaxies.}
\item {Contribute to low mass NS population and anomalous type Ia supernovae.}
\item {Look like H/He depleted AGB stars.}
\item {Clear mass cutoffs for compositions of stable final states predicted, which make for excellent tests of physics.}
\item {Future: Find one! They are not that rare and should be easy to see\ldots}
\end{itemize}
\end{frame}

% References
\begin{frame}[allowframebreaks]
  \frametitle<presentation>{References}
  \fontsize{6}{7.2}\selectfont
  \begin{thebibliography}{10}
    \beamertemplatearticlebibitems

    \bibitem{Schwab2016}
      Schwab, J. and Quataert, E. and Kasen, D.
      \newblock The evolution and fate of super-Chandrasekhar mass white dwarf merger remnants.
      \newblock {\em MNRAS}, 463:3461--3475, 2016

    \bibitem{Calamida2008}
      Calamida, A., Corsi, C. E., Bono, G., et al.
      \newblock On the White Dwarf Cooling Sequence of the Globular Cluster $\omega$ Centauri.
      \newblock {\em The Astrophysical Journal Letters}, 673, L29, 2008
      
      \bibitem{Dan2014}
        {Dan}, M. and {Rosswog}, S. and {Brüggen}, M. and {Podsiadlowski}, P.
      \newblock{The structure and fate of white dwarf merger remnants}
      \newblock {\em MNRAS} 438, 14, 2014

     \bibitem{Dan2015}
       Dan, M., Guillochon, J., Brueggen, M., Ramirez-Ruiz, E., and Rosswog, S.
       \newblock Thermonuclear detonations ensuing white dwarf mergers
       \newblock {\em MNRAS}, 454, 4411, 2015

     \bibitem{Howell2006}
       Howell, D. A., Sullivan, M., Nugent, P.E., et al. 
       \newblock The type Ia supernova SNLS-03D3bb from a super-Chandrasekhar-mass white dwarf star.
       \newblock {\em Nature}, 443, 308, 2006 

     \bibitem{Nelemans2001a}
       Nelemans, G., Yungelson, L. R., Portegies Zwart, S. F., and Verbunt, F. 
       \newblock Population synthesis for double white dwarfs . I. Close detached systems.
       \newblock {\em A\&A}, 365, 491, 2001

     \bibitem{Nelemans2001b}
       Nelemans, G., Zwart, S. F. Portegie, Verbunt, F., and Yungelson, L. R. 
       \newblock Population synthesis for double white dwarfs II. Semi-detached systems: AM CVn stars
       \newblock {\em A\&A}, 368, 939. 2001

     \bibitem{Raghavan2010}
       Raghavan, D., McAlister, H. A., Henry, T. J., et al. 
       \newblock A Survey of Stellar Families: Multiplicity of Solar-type Stars.
       \newblock {\em ApJS}, 190, 1, 2010

     \bibitem{Schwab2012}
       Schwab, J., Shen, K. J., Quataert, E., Dan, M., and Rosswog, S.
       \newblock The viscous evolution of white dwarf merger remnants.
       \newblock {\em MNRAS}, 427, 190, 2012

 \end{thebibliography}
\end{frame}

\end{document}
